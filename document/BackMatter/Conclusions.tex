\begin{conclusions}
    La simulación y modelación en los últimos 25 años ha alcanzado gran auge e importancia, gracias a su efectividad y la gran cantidad de recursos que pueden ser ahorrados con esta. La simulación de cultivos específicamente permite estimar de manera bastante fiable el rendimiento de un cultivo.
    La presente investigación tomó como base el modelo WOFOST, encontrado en Python Crop Simulation Environment (PCSE). Este modelo permite estimar los valores de las variables biológicas como el rendimiento de los órganos de almacenamiento y la biomasa total.
    Se desarrolló una interfaz visual intuitiva que permite el trabajo con datos reales para su simulación y posterior análisis mediante el uso de tablas y gráficas. La aplicación permite al usuario simular una determinada cantidad de días, parar la simulación para cambiar algún estado de acuerdo a los datos reales y continuar con la simulación. Esto permite mejorar el modelo y obtener resultados más fiables gracias a la retroalimentación.
    Tanto el objetivo general de la investigación, así como los objetivos específicos propuestos en la misma fueron cumplidos satisfactoriamente, aunque se seguirá trabajando en futuras investigaciones.
\end{conclusions}
