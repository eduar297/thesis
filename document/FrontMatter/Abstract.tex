\begin{resumen}
	La modelación y simulación son una herramienta que permite realizar análisis de impactos tecnológicos, económicos y ambientales, evaluar estrategias productivas y predecir con gran exactitud el rendimiento de cultivos. Al estudiar procesos de forma individual, muchas veces se pasan por alto relaciones e interacciones entre estos componentes, que gracias al uso de la simulación se pueden determinar mediante la realización de experimentos que sería muy costoso en la vida real o incluso imposible. En esta investigación se vinculó la teoría con la práctica, para culminar en el desarrollo de una aplicación visual que permitió, a través de la introducción de ciertos parámetros iniciales, simular el desarrollo de un cultivo y poder analizar los resultados de dicha modelación mediante el uso de gráficas y tablas.\\
	
	\textbf{Palabras clave}: simulación, modelación, cultivo, WOFOST, PCSE.
\end{resumen}

\begin{abstract}
	Modeling and simulation is a tool that allows the analysis of technological, economic and environmental impacts, the evaluation of production strategies and the accurate prediction of crop yields. When studying processes individually, relationships and interactions between these components are often overlooked, which thanks to the use of simulation can be determined by conducting experiments that would be very costly or impossible in real life. This research linked theory with practice, culminating in the development of a visual application that allowed, through the introduction of certain initial parameters, to simulate the development of a crop and to be able to analyze the results of such modeling through the use of graphs and tables.\\
	
	\textbf{Key words}: simulation, modeling, crop, WOFOST, PCSE.
\end{abstract}